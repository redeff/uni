\documentclass{article}
\everymath{\displaystyle}
\usepackage{mathtools}
\usepackage{graphicx}
\setlength{\parindent}{0pt}
\setlength{\parskip}{10pt}
\title{Hidrostática}
\date{}
\author{}
\begin{document}
\maketitle
Va a mandar apuntes
\section{Fluidos}
Propiedades del fluido
\begin{itemize}
	\item Incomplesible
	\item Ideal
	\item Estático (los elementos del fluido se mueven, pero en una región
		acotada)
	\item Densidad ($\delta$)
	\item Peso específico
	\item Elementos del fluido (partes chiquititas con muchas moléculas que
		componen el fluido)
\end{itemize}
\section{Superficie Libre}
Son todos los elementos del fluido, en contacto con la superficie

\section{Superficie Normal}
Todas las partículas de un mismo fluido, en una misma superficie de nivel, y
que se encuentran interconectado (arc-connected),
tienen la misma presión hidrostática

\section{Presión}
Para todo punto del fluido, la presión en ese punto está dado por \(
	P = h \delta g
\), donde $g$ es la gravedad, $\delta$ la densidad y $h$ la distancia desde
el punto hasta la superficie del fluido

(En realidad da \(
	P_a - P_b = \int_a^b \delta _x \cdot g _x\; dx
\) para puntos $a$ y $b$ de fluido y una path entre los dos)

\(
[P] = m \cdot \frac{kg}{m^3} \cdot \frac{m}{s^2} = \frac{kg}{ms^2} =
\frac{N}{m^2} = Pa
\)

\subsection{Presión Atmosférica}
Como el peso de los elementos de fluido de la atmósfera es muy poco,
asumimos que la presión atmosférica es igual a cualquier altura.

\section{Tornichelli}
\section{Vasos Comunicantes}
\section{Gas Ideal}
\(
	PV = nRT
\), donde $P$ es la presión, $V$ es el volumen, $n$ es la cantidad de moles
de gas, $T$ es la temperatura absoluta, y $R = 8.31446 \;
J\,K^{-1}\,mol^{-1}$

\section{Principio de Arquímedes}
El volumen de líquido desplazado por un objeto sumergido (volumen
desalojado) es el igual al volumen del objeto

Sobre todo objeto (total o parcialmente) sumergido, el líquido ejerce una
fuerza de empuje, que es igual al peso del volumen de líquido desalojado.
Tenemos $E = mg = V\delta g$, done $V$ es el volumen sumergido y $\delta$ es
la densidad del fluido

Si la densidad del objeto es igual a la del fluído, éste flota iff está todo
sumergido
\end{document}
