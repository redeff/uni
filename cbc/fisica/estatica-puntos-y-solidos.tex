\documentclass{article}
\everymath{\displaystyle}
\usepackage{mathtools}
\setlength{\parindent}{0pt}
\setlength{\parskip}{20pt}
% \setlength{\linespread}{3pt}
\linespread{1.1}
\title{Estática de cuerpos puntuales y cuerpos sólidos}
\date{}
\begin{document}
\maketitle
\section{Cuerpos Puntuales}
Se marca un punto fijo del objeto, y se presta atención sólo a ese punto.
Cuando se trabajan con puntos vale que $\sum \overline{F} = m\overline{a}$

\section{Cuerpos Sólidos Extensos (1D y 2D)}
Tienen las siguientes propiedades:
\begin{itemize}
	\item Extensión: tienen más de un punto
	\item Solidez: se pueden elegir y marcar dos puntos cualesquiera del
		cuerpo, y al medir la distancia entre ellos, es siempre la misma
		a lo largo del tiempo y bajo cualquier tipo de movimiento
	\item Masa homogénea
\end{itemize}

Todo punto $P$ en el sólido tiene una velocidad dada por:
$\overline{V_P} = \overline{\omega} \times \overline{R} + \overline{V_O}$,
donde $O$ es el centro de masa, $\omega$ es el vector momento angular, y $R$
es el vector hacia el centro ($\overline{R} = P - O$)

Se cumple también que, si no hay rotación, entonces $\sum M_F = 0$, donde
$M_F$ es el momento de la fuerza $F$, dado por $M_F = F \times (P - O)$, con
$P$ punto de incidencia de la fuerza

\begin{itemize}
	\item $\sum F = 0 \iff $ No hay desplazamiento
	\item $\sum M_F = 0 \iff $ No hay rotación
\end{itemize}

Entonces un objeto se mantiene estático sí y sólo sí $\sum F = 0$ y
$\sum M_F = 0$

\subsection{Centro de Momento}
Un centro de momento $O$ en un objeto extenso es un punto a partir del cual
se calculan los momentos de las fuerzas, y por el cual se hace de cuenta que
el objeto va a girar

El momento de una fuerza $F$ incidente en un punto $P$ con respecto a un
centro de momento $O$ se calcula como $M_F = \pm |F|\cdot d$. El signo se
define viendo si $F$ va horario ($-$) o antihorario ($+$) con respecto a
$O$, y $d$ es la distancia entre $O$ y la recta por $P$ con dirección $F$.
Las unidades del momento son $[M_F] = N\cdot m$.

Si la fuerza se aplica de forma alineada con el centro, entonces el momento
es cero (si podés elegir $O$, sirve para sacarte momentos de encima)

\subsection{Ejemplo: Tabla horizontal balanceada}
Si se tiene una tabla de cierta longitud, y se le aplican pesos conocidos a
ambos extremos, y se quiere balancear horizontalmente la tabla agregándole
un punto de vínculo, tenemos que plantear la ecuación con momentos:

\[
	\sum M_F = 0
\]

Para no tener que calcular el momento de la fuerza de vínculo, tomamos el
punto de vínculo como centro del momento. Nos queda el sistema
\[
	\begin{cases}
		\; F_1 \cdot d_1 - F_2 \cdot d_2 & = 0 \quad\text{Suma de los momentos} \\
		\; d_1 + d_2 & = d \quad\text{Longitud total de la tabla}
	\end{cases}
\]
Donde $F_i$s son las fuerzas aplicadas y $d_i$s son las distancias hasta el
punto de vínculo, y $d$ es la longitud de la tabla, que es un sistema fácil
de resolver

\subsection{Ejemplo: Tabla inclinada balanceada}
Se tiene una barra inclinada con dos objetos en las puntas. Para que sea un
sist. estático, la fuerza que ejercen los objetos en la barra es igual a la
fuerza peso, luego las fuerzas ejercidas en los extremos de la barra son de
nuevo verticales. Si volvemos a plantear el sistema obtenemos:
\[
	\begin{cases}
		\; \cos \alpha \cdot F_1 \cdot d_1 - \cos \alpha \cdot F_2 \cdot d_2 & = 0
		\quad\text{Con $\alpha$ el ángulo con la horizontal} \\
		\; d_1 + d_2 & = d \quad\text{Longitud total de la tabla}
	\end{cases}
\]
Que es equivalente al sistema anterior, ya que los cosenos se tachan pq
estan igualados a un $0$
\end{document}
