\documentclass{article}
\everymath{\displaystyle}
\usepackage{mathtools}
\title{Ejercicios}
\setlength{\parindent}{0pt}
\setlength{\parskip}{10pt}
\begin{document}
\maketitle
\section{Ejercicio 17}
Un objeto cúbico de dimensiones $L = 0.6m$ de lado cuyo peso es $P = 4450N$
está suspendido es un líquido con densidad $\delta = 944\frac{kg}{m^3}$
A distancia $\frac{L}{2}$ de la superficie del líquido

Encuentre los módulos de:
\begin{enumerate}
	\item La fuerza que recibe el objeto sobre la cara superior
	\item La fuerza total hacia arriba en el fondo del objeto
	\item Tensión en el alambre
	\item El empuje sobre el cuerpo
\end{enumerate}

\[
V = L^3 = (0.6)^3m^3 = 0.216
\]
\[
E = \delta V g = 944 \cdot 0.216 \cdot 10 \cdot N = 2039 N
\]
\[
	T = P - E = 2411N
\]

La fuerza sobre la cara superior está dada por $P\cdot A$, la presión es $P
= P_{atm} + h\cdot \delta \cdot g$. Hacemos la cuenta en la cara de arriba y
la de abajo y sale.
\end{document}
