\documentclass{article}
\everymath{\displaystyle}
\usepackage{mathtools}
\setlength{\parindent}{0pt}
\setlength{\parskip}{15pt}
\title{Polinomios}
\date{}
\begin{document}
\maketitle
Un polinomio es una combinacion lineal finita entre $1$, $x$, $x^2$, $\dots$,
$x^n$, $\dots$, es decir, todos losde la forma:
\(
P(x) = a_0x^0 + a_1x^1 + \dots + a_nx
\), donde $a_i$ son los coeficientes.

Los polinomios $P$ tales que sus coeficientes pertenezcan a un conjunto $V$
se denominan $V[x]$. Tenemos por ejemplo $R[x]$, $Q[x]$, $C[x]$, etc.
Generalmente no trabajamos con $Z[x]$, porque $Z$ no es un cuerpo

Notemos que suma y producto por escalar $v \in V$ están cerradas en $V[x]$

Si $P(x) = \sum a_ix^i$ y $a_n \neq 0$, luego el grado $\deg P$ es $n$, y
el coeficiente principal es $a_n$. El polinomio nulo $P(x) = 0$ no tiene
grado.

Tenemos $\deg (P+Q) \leq \max \{\deg P, \deg Q\}$

\section{Producto}

El producto entre polinomios se hace usando la distributiva. Se cumple
además que $\deg P\cdot Q =\deg P + \deg Q$, con $P, Q \neq 0$

\section{Igualdad}

Dos poly son iguales iff son iguales coeficiente a coeficiente

\section{Especialización}
Si $\sum a_ix^i = P \in K[x]$, y $z \in K$, luego la especialización de $P$
en $z$ es \[P(z) = \sum a_iz^i \in K\]

\section{Igualdad dado Igualdad de especializaciones}
Si dos polinomios $P, Q\in K[x]$ y existen $a_0, \dots, a_{n+1}$ distintos
tales que $P(a_i) = Q(a_i) \; \forall i$, luego $P = Q$

\section{Raíces}
Si $P\in K[x]$, $P(z) = 0$ se dice que $z$ es raíz de $P$. Una raíz
puede ``írsete'' de $K$

\section{División de Polys}
\fbox{
\parbox{\textwidth}{
Dados dos poly $P, Q \in K[x]$, $Q \neq 0$, luego existen únicos poly $C, R
\in K$, tales que
\[P = Q \cdot C + R\]

con $\deg R < \deg Q$ (o $R = 0$)
}}

La división se la computa igual que en enteros, ordenando los dos polys (P y
Q) y completando los espacios vaíos que faltan

Sabemos también que si $\phi_i$ son las raíces de $Q$, luego por la fórmula
del cociente tenemos $P(\phi_i) = R(\phi_i) \;\forall i$. Si $Q$ tiene $\deg
Q$ raíces distintas, luego esa información alcanza para resolver el sistema
y sacar $R$ (Ya que $\deg R < \deg Q$)

\subsection{Divisibilidad de Polinomios}
Decimos que $P\mid Q \iff \exists R : Q = P \cdot R$

\subsection{Teo del Resto}
Si $P \in K[x]$, y $Q(x) = x - r$, luego el resto de la división
$P(x) \div Q(x)$ es un polinomio constante, y esa constante es $P(r)$

Esto implica que $r$ es raíz de $P$ iff $(x - r) \mid P(x)$. Es decir, si
encuentro una raíz de $P$, lo puedo separar en un factor de grado $1$ y uno
de grado $\deg P - 1$

\subsection{Ruffini}
Si $P(x) = \sum^n a_ix^i$, y $P(r) = 0$, luego se puede hallar el polinomio
\[
\sum ^{n-1} b_ix^i = Q(x) = \frac{P(x)}{(x - r)}
\]
haciendo la recurrencia \(
\begin{cases}
b_n = 0 \\
b_{i-1} = a_i + r \cdot b_{i}
\end{cases}
\)

$b_{-1}$ debería ser $0$, osinó $r$ no era raíz

\section{Teorema Fundamental del Álgebra}
Para todo $P \in C[x]$, con $\deg P \geq 1$, existe $z \in C$ tal que $P(z)
= 0$. Aplicando recursivamente, sabemos que existen $\varphi_1, \dots,
\varphi_{n}$ Tal que $\varphi_i$ son raíces de $P$ y
\[P(x) = a_n \cdot \prod (x - \varphi_i)\]
Esta es la factorización en $C[x]$

\section {Rational Roots Theorem (Teo de Gauss)}
Si $P \in Z[x]$, con coeficiente principal distinto de 0,
luego para toda raíz racional de $P$ irreducible
$\frac{\alpha}{\beta}$, luego $\alpha \mid a_0$ y $\beta \mid a_n$
(No necesariamente existen raíces racionales)
\end{document}
