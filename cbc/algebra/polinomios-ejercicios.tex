\documentclass{article}
\everymath{\displaystyle}
\usepackage{mathtools}
\setlength{\parskip}{20pt}
\setlength{\parindent}{0pt}
\begin{document}
\section{Ejercicio 3}
Hallar todas las raíces de $P(x) = x^4 + 3x^3 + 4x^2+3x+10$ sabiendo que
alguna de ellas es también raíz de $Q(x) = 2x^3-2x^2-2x-4$.

Para resolver sacamos el resto $R$ de la división $\frac{P}{Q}$, y tenemos
que $P = Q\cdot C + R$ para algún poly $C$, luego toda raíz común de $P$ y
$Q$ es también de $R$

Tenemos $R(x) = 9x^2 + 9x + 18 = 9(x^2+x+2)$, que tiene raíces $\frac{-1 \pm
i\sqrt{7}}{2}$

Luego como alguna raíz de $P$ es raíz de $R$, sabemos que las dos lo son ya
que tiene coeficientes reales. tenemos:
\(
	P(x) = (x^2 + x + 2) \cdot S(x)
\)
Para algún $S$, dividimos para hallar $S$ y estamos.
\end{document}
