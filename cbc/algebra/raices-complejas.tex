\documentclass{article}
\everymath{\displaystyle}
\usepackage{mathtools}
\usepackage{amssymb}
\def\R{\mathbb{R}}
\def\C{\mathbb{C}}
\begin{document}
\section{Especializaciones de conjugados}
tenemos \(
\overline{P(r)} = P(\overline{r})
\)
Para todo $P \in \R[x], r\in \C$
\subsection{Demo}
Tenemos \(
P(r) = \sum a_ir^i
\iff
\overline{P(r)} = \sum \overline{a_ir^i} = \sum a_i\overline{r}^i =
P(\overline{r})
\)
\section{Raíces complejas de polinomios en $\R[x]$}
Para todo $P \in \R[x]$, luego $x \in \C$ es raíz de $P$ sí y solo sí
$\overline{z}$ es raíz de $P$
\subsection{Demo}
Tenemos $P(r) = 0 \iff P(\overline{r}) = \overline{0} = 0$
\subsection{Corolario}
Todo poly en $\R[x]$ se puede factorizar en factores de grado a lo sumo $2$
\subsection{Factorizaciones en $\R[x]$}
Si tenemos un polinomio $P \in \R[x]$ con raíces complejas $z \neq
\overline{z}$, luego en la factorización en $\C[x]$ tenemos los factores \(
(x - z)\cdot (x - \overline{z}) = (x^2 - 2\cdot \Re z + |z|^2)
\), que es un factor cuadrático irreducible en $\R[x]$

\section{Raíces repetidas}
Un polinomio $P \in \C[x]$ de grado $n$ puede tener menos de $n$ raíces,
pero en su factorización en $C[x]$ necesariamente tiene exactamente
$n$ factores lineales, ya que cada raíz puede aparecer varias veces

Por ejemeplo $P = (x^4 - 1)^2(x^2 - 3x + 2)(x^2-4)(x-3)^5$ tiene $7$ raíces:
$P(\pm 1, \pm i, \pm 2, 3) = 0$, pero su factorizción es: \[P(x) =
(x-1)^3(x+1)^2(x-i)^2(x+i)^2(x-2)^2(x+2)(x-3)^5\]

\subsection{Multiplicidad de una raíz}
Dado $P \in \C[x]$, $a\in \C$, decimos
\[mult(a, P) = \max \; \{k : (x-a)^k \mid P(x)\}\]

\subsubsection{En $\R[x]$}

Si $P \in R[x]$, luego $mult(z, P) = mult(\overline{z}, P) \quad \forall z \in C$

\subsection{Raíces cuadradas de complejos}
Si tenemos la ecuación de la forma $w^2 = z$, va a tener dos soluciones, y
se pueden encontrar haciendo $w = c + di$, $z = a + bi$, tentonces tenemos

\[
	(c + di)(c + di) = a + bi
\]
\[
	c^2 - d^2 + 2cdi +  = a + bi
\]

Además $|w|^2 = z$, luego tenemos:
\[
	c^2 + d^2 = |z|
\]

En total tenemos entonces
\[
\begin{cases}
	c^2+d^2 & = |z| \\
	c^2 - d^2 & = \Re z \\
	2cd & = \Im z
\end{cases}
\]

Con las primeras dos podemos sacar $c^2$ y $d^2$, y con la tercer ecuación
sacás qué signos poner (va a haber dos posibilidades)

\subsubsection{Forma trigonométrica}
También podemos plantear $|w| = \sqrt{|z|}$ y $\arg w = \frac{\arg z}{2}$,
que alcanza para determinar $w$, pero los ángulos quedan feos
\end{document}
