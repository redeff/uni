\documentclass{article}
\everymath{\displaystyle}
\usepackage{mathtools}
\setlength{\parskip}{20pt}
\setlength{\parindent}{0pt}
\title{Relación entre los Coeficientes y sus Raíces}
\date{}
\begin{document}
\maketitle
\section{Para cuadráticos}
Tenemos \[(x - r_1)(x - r_2) = x^2 - (r_1+r_2)x + r_1r_2\]
Luego el opuesto del coeficiente lineal es la suma de las raíces, y el
independiente es el producto de las raíces.

\section{Para cúbicos}
Tenemos \[
	(x - r_1)(x - r_2)(x - r_3) = x^3 - x^2\sum r_i + x\sum r_ir_j -
	r_1r_2r_3
	\]

\section{En general}
\[
	\prod^n (x - r_i) = \sum_k^n \left((-1)^kx^k\sum_{S \subseteq \{1\dots
			n\}}^{|S| = n-k} \left(\prod_{i\
\in S} r_i\right)\right)
\]
Es decir,
\[
	\frac{a_k}{a_n} = (-1)^k \sum_{S \subseteq \{1\dots
			n\}}^{|S| = n-k} \left(\prod_{i\
\in S} r_i\right)
\]
\section{Calcular polinomios simétricos}
Si $a\;b\;c$ son raíces de un polinomio cúbico, tenemos:
\[a^2+b^2+c^2 = (a+b+c)^2 - 2(ab + bc + ca) = -\left(\frac{a_2}{a_3}\right)^2 -
2\left(\frac{a_1}{a_3}\right)\]

\[\frac{1}{a} + \frac{1}{b} + \frac{1}{c} = \frac{ab + bc + ca}{abc} =
	\frac{\frac{a_2}{a_3}}{-\frac{a_0}{a_3}} = -\frac{a_2}{a_0}
\]
\end{document}
