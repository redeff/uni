\documentclass{article}
\everymath{\displaystyle}
\usepackage{mathtools}
\setlength{\parindent}{0pt}
\setlength{\parskip}{10pt}
\begin{document}
\section{Ejercicio 1}
Sean $B = \{(1 \; 0\; -1), (2 \; 1 \; 0), (3 \; 0 \; 0)\}$ y
$B = \{(0 \; 1\; -1), (-1 \; 0 \; 0), (0 \; 1 \; 0)\}$, y tenemos \(
f : R^3 \to R^3 \quad M_{BB^{'} }
\)

Tenemos $C_{B'B} = ((B'_i)_B \; \mid \; \dots)$, porque queremos un
$C_{B'B}$ tal que \[C_{B'B} \times (v)_{B'} = (v)_B\] para todo $v$
En particular si tomo $v_i = B'_i$, tenemos \[
	C_{B'B} \times I = ((B'_i)_B \; \mid \; \dots)
\]
\end{document}
