\documentclass{article}
\everymath{\displaystyle}
\usepackage{mathtools}
\usepackage{amssymb}
\def\R{\mathbb{R}}
\def\C{\mathbb{C}}
\def\Z{\mathbb{Z}}
\def\V{\mathbb{V}}
\def\U{\mathbb{U}}
\setlength{\parskip}{20pt}
\setlength{\parindent}{0pt}
\title{Autovalores y Autovectores (Eigenvalues and Eigenvectors) de TLs}
\date{}
\author{}
\begin{document}
\maketitle
\section{Definición de Autovectores y Autovalores}
dado \(
	f : \V \to \V
\) TL, Para todo \(
v \in \V_{\neq 0}, \lambda \in \R : fv = \lambda \cdot v
\), decimos que $v$ es un autovector de $f$, con autovalor asociado
$\lambda$

\section{Hallar Autovalores}
Si tenemos $Mf$, con $f : \V \to \V$, nos queda resolver la ecuación
\[
	Mf\cdot v = \lambda \cdot v
\]
\[
	(Mf - I\lambda)\cdot v = 0_\V
\]
Que tiene soluciones no triviales iff $\det \;(Mf - \lambda I) = 0$
\subsection{Polinomio caracerístico}
Dado $f : \V \to \V$
se dice $p_f \; \lambda = \det \; (Mf - \lambda I)$.
Este polinomio en $\lambda$ se llama el polinomio caracerístico

\subsubsection{Prop de $p_f\; \lambda$}
\begin{itemize}
	\item $\lambda$ es autovalor iff $\lambda$ es raíz real de $p_f \; \lambda$
	\item $p_f \; \lambda \in \R[\lambda]$
	\item Se define $S_\lambda = \{v : fv = \lambda v\}$ al subespacio
		asociado, o autoespacio
	\item $\deg p_f = \dim \V$
	\item el coeficiente principal de $p_f$ es $(-1)^{Dim \V}$
\end{itemize}

\section{Hallar Autovectores}
Dado $f : \V \to \V$,
con el métrodo anterior hallamos $\lambda \in \R$ tal que $p_f \; \lambda = 0$,
entonces para hallar los autovectores asociados sólo tenemos que resolver el
sistema homogéneo \[
	(Mf - I\cdot \lambda)\cdot v = 0_\V
\]
Sabiendo que la matriz $(Mf - I\cdot \lambda)$ no es inversible, los $v$s
que cumplen forman un subespacio no nulo

\section{Autovalores y Autovectores de una matriz}
Dado $A \in \R^{n \times n}$, se hacen las mismas definiciones que para la
transformación lineal $f : \R^n \to \R^n, fv = A\cdot v$. Se tiene: \[
	p_A \; \lambda = \det \; (A - \lambda I) = |A - \lambda I|
	\] y \[
	S_\lambda = \{v \in \R^n : A\cdot v = \lambda \cdot v\}
\]

\subsection{Propiedades}
\begin{enumerate}
	\item $\det A = 0 \iff p_A \; 0 = 0$, y tenemos $S_0 = Nu \; f$
\end{enumerate}

\section{Diagonalizabilidad}
Una transformación lineal $f : \V \to \V$ es diagonalizable iff
existe una base de $\V$ formada por autovectores de $f$

Una definición equivalente es:

Para transformaciones lineales, $f$ es diagonalizable si $M_B \;f $ es
diagonal para alguna base $B_\V$ formada por los autovalores de $f$,
y donde el valor en cada elemento de la diagonal es el autovalor asociado al
autovector de esa columna

Tenemos $Mf = C_{BE} \cdot M_B f \cdot C_{EB} = C_{BE} \cdot M_B f \cdot
(C_{BE})^{-1}$

Para matrices, $A$ es digonalizable si $A \sim D$ para alguna diagonal $D$,
es decir, si existen $D$ diagonal y $C$ inversible
tal que $A = C\cdot D \cdot C^{-1}$.
Si esto pasa, luego necesariamente, tendremos
\(
D = \begin{bmatrix}
	\lambda_1 & \dots & 0 \\
	\vdots & \ddots & \vdots \\
	0 & \dots & \lambda_n \\
\end{bmatrix}
\), y \(
C = (v_i | \dots)
\) con $v_i$ los autovectores y $\lambda_i$ sus autovalores asociados
\subsection{Propiedad}
\begin{itemize}
	\item $S_\lambda \cap S_\mu = 0$ para $\lambda \neq \mu$
	\item $\dim S_\lambda \geq 0$ Para todo autovalor $\lambda$
	\item $\dim S_\lambda \leq mult_\lambda \; p_A$
	\item $A$ es diagonalizable $\iff$ todas las raíces de $p_A$ son reales
		y $\dim S_\lambda = mult_\lambda \; p_A$ para toda raíz $\lambda$ de
		$p_A$. Para ver si esto es verdad alcanza con ver los rangos de
		todas las matrices de la forma $Rg \;(A - I\lambda)$ con $\lambda$
		autovalor
\end{itemize}
\subsubsection{Demo de $S_\lambda \cap S_\mu = 0$}
Por absurdo, asumamos $0 \neq v \in S_\lambda \cap S_\mu$, entonces tenemos
$A\cdot v = \lambda \cdot v = \mu \cdot v$, entonces $\lambda = \mu$.
\subsection{Potencias de matrices diagonalizables}
Si tenemos una matriz $A \in \R^{n\times n}$ escrita como $A = CDC^{-1}$ con
$C$ inversible y $D$ diagonal, luego $A^n = CD^nC^{-1}$, para todo $n \in \Z$

Elevar una matriz diagonal a una potencia es fácil, ya que alcanza con
elevar cada elemento de la diagonal

\subsubsection{Polinomios (y series polinómicas)}
Si tenemos $fx = \sum a_ix^i$, podemos definir $fA = \sum a_iA^i = \sum
a_iCD^iC^{-1} = C \left(\sum a_iD^i\right) C^{-1}$ = \(
	C \cdot \begin{bmatrix}
		f\; d_1 & 0 & \dots & 0 \\
		0 & f \; d_2 & \dots & 0 \\
		\vdots & \vdots & \ddots & \vdots \\
		0 & 0 & \dots & f \; d_n \\
	\end{bmatrix} \cdot C^{-1}
\)

\subsection{Autovalores de potencias de $A$}
Si $\lambda$ es autovalor de $A$, luego $\lambda^2$ es autovalor de $A^2$
Asimismo, si $v$ es autovector de $A$ con autovalor $\lambda$, tenemos que
$v$ es autovector de $A^2$ con autovalor $\lambda^2$, luego $A$
diagonalizable, luego $A^2$ diagonalizable.

\section{Autovalores bajo Cambios de Base}
Si los autovalores de $Mf$ son los mismos que los de $M_Bf$ para cualquier
base $B$ de $\V$, y si $v$ es autovector correspondiente a $\lambda$ en
$Mf$, luego $v_B$ es autovector correspondiente a $\lambda$ de la matrix
$M_Bf$
\end{document}
