\documentclass{article}
\title{Integrales}
\date{}
\author{}
\everymath{\displaystyle}
\usepackage{mathtools}
\begin{document}
\maketitle
\section{Teorema Fundamental de Cálculo}
Sea \(
F(x) = \int_a^xf(t) \; dt
\), luego: \(
	f
\) contúnua implica \(
F'(x) = f(x)
\)
\subsection{Composición}
Si tenemos por ejemplo \(
F(x) = \int_0^{g(x)}f(t) \; dt
\), luego \(
F'(x) = f(g(x)) \cdot g'(x)
\) si $f$ continua, $g$ derivable
\subsection{Partir una integral}
Tenemos \(
\int_a^b f(t)\;dt = \int_a^0 f(t)\;dt + \int_0^b f(t)\;dt
\)
\subsection{Dar vuelta una integral}
Tenemos \(
\int_a^b f(t) \; dt = -\int_b^a f(t) \; dt
\)
\subsection{Juntando todo}
\(
F(x) = \int_{s(x)}^{r(x)} f(t) \; dt
\) con $f$ cont y $s$, $r$ deriv, luego,
\[
F'(x) = f(r(x)) \cdot r(x) - f(s(x)) \cdot s'(x)
\]
\section{Def Primitivas}
$G(x)$ es una primitiva de $f(x)$ si $G'(x) = f(x)$
\subsection{Todas las primitivas}
Si $G(x)$ es una primitiva de $f(x)$, luego todas las primitivas de $f(x)$
son de la forma $G(x) + c$ para algún $c\in R$
\section{Integrales definidas}
Si $G'(x) = f$ y $f$ continua, luego la integral
\(
A(x) = \int_a^xf(t) \; dt
\) se puede escribir como $A(x) = G(x) + c$ para algún $c$, además sabemos
$A(a) = 0$, luego $c = -G(a)$. Tenemos entonces:
\subsection{Regla de Barrow}
\[
	A(b) = \int_a^b f(b) \; dt = G(b) - G(a) \text{ Para $G$ primitiva de $f$}
\]
Esa integral se llama integral definida, cuando los extremos son números
\subsection{Notación}
Se puede escribir \(
\left. \int_a^b f'(x)\; dx = f(x)\right| _a^b
\)
\section{Integral Indefinida}
Se nota \(
\int f'(x) \; dx = f(x) + c
\) para describir todas las primitivas
\subsection{Tabla de primitivas}
\[
	x^n \to \frac{x^{n+1}}{n+1} \text{ Si $x \neq -1$}
\]

\[
	x^{-1} \to \ln |x|
\]

\[
	e^x \to e^x
\]

\section{Métodos para calcular integrales}
\subsection{Por sustitución}
Por regla de la cadena, tenemos: \(
\int f'(g(x)) \cdot g'(x) = f \circ g
\)
\subsubsection{Ejemplo}
Queremos calcular \(
\int 2x \cdot \cos(x^2 +1 )
\). Tomamos $fx = \sin x$, $g x = x^2 + 1$, luego por regla de la cadena:
\[
	\int 2x \cdot \cos(x^2 +1 ) = \int f'(g(x)) \cdot g'(x) = f \circ g =
	\sin (x^2 + 1)
\]
\subsubsection{Notación}
El ejercicio anterior se puede hacer un cambio de variable $u = x^2 + 1$, y
se tiene $u'(x) = 2x \iff du = 2x \; dx$, entonces la integral queda

\[
	\int 2x \cdot \cos(x^2 +1 ) = \int du \cdot \cos u = \sin u = \sin
	(x^2+1)
\]

\subsubsection{En general para divisiones tenemos}
Si tenemos \(
\int \frac{f'(x)}{f(x)} \; dx
\), luego tomemos $u = f(x)$, tenemos $du = f'(x) \; dx$, entonces
\(
\int \frac{f'(x)}{f(x)} \; dx = \int \frac{1}{u} \; du = \ln |u| = \ln |f(x)|
\)

\subsubsection{Y para productos}
Si tenemos \(
\int f'(x)\cdot f(x) \; dx
\), luego tomemos $u = f(x)$, tenemos $du = f'(x) \; dx$, entonces
\(
\int f'(x) \cdot f(x) \; dx = \int u \; du = \frac{1}{2}u^2 =
\frac{1}{2}f(u)^2
\)

\end{document}
