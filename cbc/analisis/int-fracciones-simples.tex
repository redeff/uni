\documentclass{article}
\everymath{\displaystyle}
\usepackage{mathtools}
\begin{document}
\section{Fracciones Simples}
Sabemos:
\(
\int \frac{A}{x - c} = A \ln |x - c|
\), entonces vamos a tratar de escribir cosas como suma de términos de esa
pinta

Si tenemos una función racional con grado de abajo mayor al de arriva, y con
el polinomio de abajo con raices todas distintas y reales, entonces el sist.
lineal siempre tiene solución.

\subsection{$\frac{1}{(x-1)(x+3)}$}
\[
\int \frac{1}{(x-1)(x+3)} = \int \frac{\frac{1}{4}}{x-1} +
\frac{-\frac{1}{4}}{x+3} = 
\]
\[
\frac{1}{4} \ln |x-1| - \frac{1}{4} \ln |x+3|
\]

Para sacar los coeficientes que van arriba, planteamos las ecuaciones e igualamos los polinomios término a término.

\subsection{División de polinomios}
Sólo se puede hacer ese método cuando el polinomio de arriva tiene grado
menor al de abajo, entonces se necesita hacer división de polinomios.

Si $P(x)$ y $Q(x) \neq 0$ son polinomios, luego existen únicos $C(x)$ y
$R(x)$ tal que: \(
P(x) = Q(x) \cdot C(x) + R(x)
\) Con \(
	\deg R < \deg Q
\)

\subsection{$\frac{x^3+2x}{x^2 - 1}$}
\(
\int \frac{x^3 + 2x}{x^2 - 1} = \int x + \int \frac{3x}{x^2-1}
\), que lo sabemos hacer
\end{document}
