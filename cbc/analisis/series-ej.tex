\documentclass{article}
\everymath{\displaystyle}
\usepackage{mathtools}
\setlength{\parindent}{0pt}
\setlength{\parskip}{10pt}
\begin{document}
\section{Ejercicio}
Analizar la convergencia de $\sum_i^{\infty}
\frac{(2x-3)^n}{5^n}\left(\frac{\ln (n+5)}{n+4}\right)$

Como \(
\lim \left(\frac{\ln(n+5)}{n+4}\right)^\frac{1}{n} = 1
\), si $\left|\frac{2x-3}{5}\right| \neq 1$, luego por cauchy es fácil
analizar la convergencia.

Dps hay que analizar los casos $x = 4$ y $x = -1$
\end{document}
