\documentclass{article}
\everymath{\displaystyle}
\usepackage{mathtools}
\begin{document}
\section{Área entre curvas}
Si tenemos dos funciones $f : [a; b] \to R$ y $g : [a; b] \to R$ y
queremos calcular el área encerrada pos las dos curvas en ese intervalo.
Hay que hallar los puntos de intersección entre $f$ y $g$, luego ver en cada
intervalo cuál de $f$ y $g$ juega de techo y cual de piso, y finalmente
planear en cada intervalo \(
\int_{a_i}^{b_i} t(x) - p(x)
\), donde $[a_i; b_i]$ es el intervalo entre las raíces y $t$ y $p$ son la
función piso y la techo resp.

\subsection{Ejemplo}
Tenemos las curvas $y = x \ln x$, $y = 0$, $x = e$.Tenemos que intersectar
$x \ln x = 0$ y nos da $x = 1$, entonces queres \(
\int_1^e (x \ln x - 0) \; dx = \left.\frac{1}{2}x^2\ln x\right|_1^e -
	\frac{1}{2}\int_1^e x = \frac{e^2 + 1}{4}
\)
\end{document}
